
\definecolor{darkgreen}{rgb}{0.0,0.4,0.0}
\definecolor{darkblue}{rgb}{0.0,0.0,0.4}
\definecolor{lightblue}{rgb}{0.3,0.3,0.7}

\newcommand{\getbox}[1]{
	{\color{white}\hrule height 15pt}
	\begin{tcolorbox}[title=\textbf\mt{Erőforrás}{},colback=lightgray,colframe=darkgreen,left=0pt,right=0pt,top=1pt,bottom=1pt,boxsep=5pt,arc=0mm,boxrule=5pt,nobeforeafter,sharp corners=all]
		\ttfamily\textbf{GET #1}
	\end{tcolorbox}
	{\color{white}\hrule height 15pt}
}

\newcommand{\postbox}[1]{
	{\color{white}\hrule height 15pt}
	\begin{tcolorbox}[title=\textbf\mt{Erőforrás}{},colback=lightgray,colframe=darkblue,left=0pt,right=0pt,top=1pt,bottom=1pt,boxsep=5pt,arc=0mm,boxrule=5pt,nobeforeafter,sharp corners=all]
		\ttfamily\textbf{POST #1}
	\end{tcolorbox}
	{\color{white}\hrule height 15pt}
}

\newcommand{\putbox}[1]{
	{\color{white}\hrule height 15pt}
	\begin{tcolorbox}[title=\textbf\mt{Erőforrás}{},colback=lightgray,colframe=lightblue,left=0pt,right=0pt,top=1pt,bottom=1pt,boxsep=5pt,arc=0mm,boxrule=5pt,nobeforeafter,sharp corners=all]
		\ttfamily\textbf{PUT #1}
	\end{tcolorbox}
	{\color{white}\hrule height 15pt}
}

\newcommand{\deletebox}[1]{
	{\color{white}\hrule height 15pt}
	\begin{tcolorbox}[title=\textbf\mt{Erőforrás}{},colback=lightgray,colframe=darkred,left=0pt,right=0pt,top=1pt,bottom=1pt,boxsep=5pt,arc=0mm,boxrule=5pt,nobeforeafter,sharp corners=all]
		\ttfamily\textbf{DELETE #1}
	\end{tcolorbox}
	{\color{white}\hrule height 15pt}
}

\newcommand{\apiinlineitem}[2]{{\textbf{#1}} #2}

\newcommand{\apiitem}[2]{
	\block{
		\subsubsection{#1}
		
		\p{#2}
	}\\
}

\newcommand{\param}[1]{{\color{gray}\textbf{\textit{{#1}}}}}

\newcommand{\bearer}{
	\apiitem{\mt{Fejléc}{Header}}{
		\begin{description}
			\item[Authorization] Bearer token
		\end{description}
	}
}

\newcommand{\requestHeader}[1]{\apiitem{\mt{Kérés fejléce}{Request header}}{#1}}

\newcommand{\responseHeader}[1]{\apiitem{\mt{Válasz fejléce}{Response header}}{#1}}

\newcommand{\token}{\apiinlineitem{Authorization}{Bearer token}}

\newcommand{\contentJson}{\apiinlineitem{Content-Type}{application/json}}

\newcommand{\request}[1]{\apiitem{\mt{Kérés}{Request}}{#1}}

O\newcommand{\responseCodes}[1]{\apiitem{\mt{Válasz kódok}{Response codes}}{#1}}

\newcommand{\response}[1]{\apiitem{\mt{Válasz}{Response}}{#1}}

\newcommand{\method}[1]{\apiinlineitem{\mt{Metódus}{Method}}{#1}}

\newcommand{\apiurl}[1]{\apiinlineitem{\mt{Url}{Url}}{#1}}

\newcommand{\params}[1]{\apiitem{\mt{Paraméterek}{Parameters}}{#1}}


%response codes

\newcommand{\ok}{\apiinlineitem{200}{OK}}

\newcommand{\forbidden}{\apiinlineitem{403}{\mt{Nincs megfelelő engedély}{Forbidden}}}

\newcommand{\unauthorized}{\apiinlineitem{401}{\mt{Nincs bejelentkezve}{Unauthorized}}}

\newcommand{\badRequest}{\apiinlineitem{400}{\mt{Hibás kérés}{Bad request}}}

\newcommand{\notFound}{\apiinlineitem{404}{\mt{Nem létezik}{Not found}}}

\newcommand{\created}{\apiinlineitem{201}{\mt{Létrehozva}{Created}}}

\newcommand{\deleted}{\apiinlineitem{204}{\mt{Törölve}{Deleted}}}

\newcommand{\modified}{\apiinlineitem{204}{\mt{Módosítva}{Modified}}}

\newcommand{\renamed}{\apiinlineitem{204}{\mt{Átnevezve}{Renamed}}}

\newcommand{\granted}{\apiinlineitem{204}{\mt{Engedély megadva}{Authorization granted}}}

\newcommand{\idConflict}{\apiinlineitem{409}{\mt{A cél azonosító már létezik.}{The target id already exists.}}}

\newcommand{\listCodes}{
	\responseCodes{
		\forbidden
		
		\unauthorized
		
		\ok
	}
}

\newcommand{\getCodes}{
	\responseCodes{
		\forbidden
		
		\unauthorized
		
		\notFound
		
		\ok
	}
}

\newcommand{\putCodes}{
	\responseCodes{
		\forbidden
		
		\unauthorized
		
		\badRequest
		
		\notFound
		
		\modified
		
		\created
	}
}

\newcommand{\moveCodes}{
	\responseCodes{
		\forbidden
		
		\unauthorized
		
		\badRequest
		
		\notFound
		
		\idConflict
		
		\renamed
	}
}

\newcommand{\deleteCodes}{
	\responseCodes{
		\forbidden
		
		\unauthorized
		
		\notFound
		
		\deleted
	}
}

%common parameters

\newcommand{\itemParam}{\apiinlineitem{item}{\mt{A cikk azonosítója}{Identifier of the item}}}
\newcommand{\userParam}{\apiinlineitem{user}{\mt{Felhasználó azonosítója}{User's identifier}}}
\newcommand{\warehouseParam}{\apiinlineitem{warehouse}{\mt{Telephely azonosítója}{Identifier of the warehouse}}}
\newcommand{\storageParam}{
	\warehouseParam
	
	\apiinlineitem{storage}{\mt{Raktár azonosítója}{Identifier of the storage}}
}
\newcommand{\paginatorParam}{
	\apiinlineitem{q}{\mt{Keresési szöveg}{}}
	
	\apiinlineitem{offset}{\mt{Visszaadott elemek kezdő pozíciója, 0 ha hiányzik}{}}
	
	\apiinlineitem{limit}{\mt{Maximum visszaadott elemek száma, 20 ha hiányzik}{}}
}
\newcommand{\paginatorParamA}{
	\apiinlineitem{q}{\mt{Keresési szöveg}{}}
	
	\apiinlineitem{archived}{\mt{Ha \code{true} akkor arhivált értékeket is visszaad}{}}
	
	\apiinlineitem{offset}{\mt{Visszaadott elemek kezdő pozíciója, 0 ha hiányzik}{}}
	
	\apiinlineitem{limit}{\mt{Maximum visszaadott elemek száma, 20 ha hiányzik}{}}
}
\newcommand{\putParam}{
	\apiinlineitem{update}{\mt{Ha \code{true} akkor meglévő erőforrást frisíthet}{Update}}
	
	\apiinlineitem{create}{\mt{Ha \code{true} akkor új erőforrást létrehozhat}{create}}
}
\newcommand{\searchParam}{
	\apiinlineitem{q}{\mt{Keresési szöveg}{}}
	
	\apiinlineitem{qwarehouse}{\mt{Ha \code{true} akkor telephely szerint is csoportosít}{}}
	
	\apiinlineitem{qstorage}{\mt{Ha \code{true} akkor raktár szerint is csoportosít}{}}
	
	\apiinlineitem{lot}{\mt{Ha \code{true} akkor lot szám szerint is csoportosít}{}}
	
	\apiinlineitem{serial}{\mt{Ha \code{true} akkor sorozatszám szerint is csoportosít}{}}
	
	\apiinlineitem{offset}{\mt{Visszaadott elemek kezdő pozíciója, 0 ha hiányzik}{}}
	
	\apiinlineitem{limit}{\mt{Maximum visszaadott elemek száma, 20 ha hiányzik}{}}
}

\newcommand*\wrapletters[1]{\wr@pletters#1\@nil}
\def\wr@pletters#1#2\@nil{#1\allowbreak\if&#2&\else\wr@pletters#2\@nil\fi}

\newcommand{\struct}[1]{\hyperlink{#1}{\code{\wrapletters{#1}}}}

\setlength{\listparindent}{0pt}
\setlength{\parindent}{0pt}
\setlength{\leftmargin}{0pt}
\setlength{\parskip}{0pt}
\offinterlineskip

\lstset{language=JavaScript}

\subimport{}{types}
\subimport{}{authentication}
\subimport{}{authorization}
\subimport{}{items}
\subimport{}{limits}
\subimport{}{operations}
\subimport{}{search}
\subimport{}{storages}
\subimport{}{units}
\subimport{}{users}
\subimport{}{warehouses}
\subimport{}{db}

