\block{
	\section{\mt{Felhasználók}{Users}}

	\subsection{\mt{Felhasználók lekérése}{Retrieval of users}}

	\getbox{/users?q=\param{q}\&offset=\param{offset}\&limit=\param{limit}}
}
\params {
	\paginatorParam
}
\requestHeader{
    \token
}
\listCodes
\responseHeader{
    \contentJson
}
\response{
	\mt{Egy \struct{User} objektum lista.}{A JSON object where the key is the identifier of the users and it's value is each users' properties.}
	
	\lstinputlisting{users.list.response.json}

}

\block{
	\subsection{\mt{Felhasználó azonosítójának változtatása}{Changing the user's identifier}}

	\postbox {/users}
}
\requestHeader{
    \token

    \contentJson
}
\request{
	\mt{Egy \struct{MoveRequest} objektum.}{Changes the user's identifier.}
	
	\lstinputlisting{users.move.request.json}
}
\moveCodes

\block{
	\subsection{\mt{Felhasználó lekérése}{Retrieval of user}}

	\getbox {/users/\param{user}}
}
\params {
	\userParam
}
\requestHeader{
    \token
}
\getCodes
\responseHeader{
    \contentJson
}
\response{
	\mt{Egy \struct{User} objektum.}{The user's properties in JSON format.}
	
	\lstinputlisting{users.get.response.json}
}

\block{
	\subsection{\mt{Felhasználók módosítása, létrehozása}{Creating, modifying users}}

	\putbox {/users/\param{user}}
}
\params {
	\userParam
}
\requestHeader{
    \token

    \contentJson
}
\request{
	\mt{Egy \struct{UserPutRequest} objektum.}{The user's properties in JSON format.}

	\lstinputlisting{users.put.request.json}
}
\putCodes

\block{
	\subsection{\mt{Felhasználó törlése}{Delete an user}}

	\deletebox {/users/\param{user}}
}
\params {
	\userParam
}

\requestHeader{
    \token
}
\deleteCodes
