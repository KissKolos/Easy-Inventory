\chapter{\mt{Projekt leírás}{}}


\block{%
	\section{\mt{Bevezetés}{}}
	
	\p{\mt{%
		A rendszer célja, lehetővé tenni cégek számára, hogy bármiféle képesítés vagy különösebb hozzáértés nélkül is egyszerűen tudják kezelni a legkomplexebb nyilvántartásokat is. A rendszer lehetővé teszi különböző jogkörök felhasználókhoz való csatolását, melyek befolyásolják, hogy a nyilvántartás mely részeit tekintheti meg és milyen műveleteket hajthat végre a programon belül az adott felhasználó. A rendszer képes telephelyek, raktárak és cikkek nyilvántartására, miközben figyelembe veszi a különböző egységek és tárolóhelyek közötti kapcsolatok kezelését is. Mindezek mellett, egyik fő szempontunk, hogy a rendszer a lehető leg helytakarékosabb módon rendezze el az adott cikkeket a raktárokon és a telephelyeken belül, figyelembe véve azt is, hogy ne kerülhessenek egymás mellé olyan anyagok, amelyek reakcióba lépése károsodásokhoz vezethet. Az alábbiakban részletesen ismertetjük a program főbb funkcióit és kezelési lehetőségeit. Ezt a feladatot azért választottuk, mert kellő kihívásnak találtuk.
	}{}}
}


\block{%
	\section{\mt{Munkamegosztás}{}}

	\subsection{Pék Gergő}

	\p{%
		Csoportvezetés, API megtervezése és elkészítése, mobil és asztali alkalmazás létrehozása, webes frontend és API szerver architektúrájának tervezése, adatbázis megtervezése és elkészítése, dokumentáció megírása, stílusának tervezése és kód minőségének ellenőrzése.
	}
}

\block{%
	\subsection{Kiss Kolos}

	\p{%
		API tesztek írása, webes frontend elkészítése, felhasználói dokumentáció megírása és nyelvi helyességének ellenőrzése, design elemek (ikonok, logo) megrajzolása, nyelvi fájlok (angol, magyar) elkészítése és asztali alkalmazás felhasználói felülete.
	}
}

\block{%
	\subsection{Garai Nándor}

	\p{%
		Webes és API tesztek írása, API szerver implementálása, web frontend design megtervezése, tesztadatok generálása, adatbázis tervezése, bemutató készítése és asztali alkalmazás felhasználói felülete.
	}
}

\block{%
	\section{Munka folyamata}

	\p{%
		A fejlesztés teszt alapú megközelítéssel zajlott. Először az API készült el, majd párhuzamosan dolgoztunk a három felületen (mobil, asztali, web) és a hozzájuk tartozó teszteken.
		Végül a dokumentáció elkészítése és a végső simítások következtek. 
	}
}

\block{%
	\section{Használt fejlesztői eszközök}

	\p{%
		\begin{itemize}
			\item Netbeans
			\item VSCode
			\item Android Studio
			\item PHPMyAdmin
			\item Inkscape
			\item XeLatex
		\end{itemize}
	}
}







