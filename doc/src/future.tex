
\chapter{\mt{Továbbfejlesztési lehetőségek}{}}

\block{
\section{\mt{Adminisztráció}{}}

\subsection{\mt{Integráció monitorozó rendszerekkel}{}}

\p{
	Integráció monitorozó rendszerekkel mint például Grafana.
	Ezen keresztül lehetne nézni a logokat.
}
}

\block{
\subsection{\mt{Központi beléptető rendszer}{}}

\p{
	A központi beléptető rendszer megkönnyítené a felhasználókezelést és segítene beintegrálni a rendszert nagyobb vállalatok rendszerébe.
}
}

\block{
\subsection{\mt{Webhook-ok}{}}

\p{
	Webhook-okat lehetne létrehozni, hogy különböző folyamatokat elindítson ha történek egy bizonyos esemény.
}
}

\block{
\section{\mt{Statisztika}{}}

\subsection{\mt{Kördiagram telephely/raktár tartalmáról}{}}

\p{
	Egy kördiagram ami mutatná a raktár tartalmát és miből milyen arányban van.
	Lehetne darabszám vagy elfoglalt hely alapján is elkészítve a diagram
}
}

\block{
\subsection{\mt{Különböző raktárak/telephelyek összehasonlítása}{}}

\p{
	Lenne egy táblázat ami összehasonlítja a raktárak vagy telephelyek tartalmát árucikkek szerint
	Könnyebb lenne átlátni és rendszerezni a raktárat
}
}

\block{
\section{\mt{Használat}{}}

\subsection{\mt{Térkép a raktárról és benne a tárgyakról}{}}

\p{
	Amikor egy konkrét tárgyra rákeresünk akkor egy térkép megjelenne és 
		lehetne látni hogy a raktáron belül hol találhatóak a tárgyak
}
}

\block{
\subsection{\mt{Rendszeres művelet  order}{}}

\p{
	Létrelehetne hozni rendszeresített műveletet ami a megadott időközönként létrehoz egy műveletet.
	A hozzá tartozó tárgyak polc számát automatikusan kiszámolná.
}
}

\block{
\subsection{\mt{Mértékegység átváltás}{}}

\p{
	Mértékegységeket át lehessen váltani, hogy a raktár telítettségét jobban meg lehessen mondani, és könnyebb legyen kezelni a raktár tárolókapacitását.
}
}

\block{
\subsection{\mt{Raktártípusok}{}}

\p{
	Raktártípusokat lehessen létrehozni és beállítani. 
	Ennek köszönhetően amikor új raktárat hozunk létre akkor az összes korlátozás és egyéb dolog automatikusan be lenne állítva.
}
}

\block{
\subsection{\mt{Riasztás beállítás}{}}

\p{
	Riasztást lehetne beállítani ha hamarosan elfogy valami.
}
}

\block{
\subsection{\mt{QR kód olvasó}{}}

\p{
	QR kód olvasó a telefonos alkalmazáshoz: A mobilalkalmazásban QR kódok olvasásának beépítése, ami egyszerűsítheti a felhasználók számára az információk gyors elérését, illetve felvitelét. 
}
}


