\block{%
	\section{\mt{Dokumentáció}{}}

	\subsection{\mt{Áttekintés}{}}
	
	\p{\mt{%
		A projekt dokumentációja \LaTeX dokumentum leíró nyelvet használja.
		A forrás fájlok a \code{doc/src} mappában találhatók.
		A legtöbb stílus elem és a csomagok importálása a \code{shared.tex} fájllban van beállítva.
	}{}}
}

\block{%
	\subsection{\mt{Felhasználói dokumentáció}{}}
	
	\p{\mt{%
		A felhasználói dokumentáció a \code{user} mappában található.
		A dokumentum által használt ikonok a \code{user/resources/icon} mappában találhatóak.
		Az ikonok megváltoztatása esetén az itt található ikonokat is frissíteni kell.
		A \code{user/resources/web} , \code{user/resources/mobile} és \code{user/resources/desktop} mappákban található képeket a programok E2E tesztjei készítik.
		A programok kinézetének változtatásakor az itteni képeket érdemes frissíteni.
	}{}}
}

\block{%
	\subsection{\mt{Kód dokumentáció}{}}
	
	\p{\mt{%
		A kód dokumentáció a \code{code} mappában találhatóak.
		Az dokumentum által használt kód részek a \code{code/generated} mappában találhatóak.
		Ezeket a kód részeket az építő program autómatikusan generálja.
	}{}}
}

\block{%
	\subsection{\mt{Teszt dokumentáció}{}}
	
	\p{\mt{%
		A teszt dokumentáció a \code{test} mappában, a dokumentum által használt kód részletek pedig a \code{test/generated} mappában találhatóak.
		Ezeket a kód részleteket az építő program autómatikusan generálja.
	}{}}
}

\block{%
	\subsection{\mt{Kód beillesztő}{}}
	
	\p{\mt{%
		A kód beillesztő program a \code{codeextractor.py} python program.
	}{}}
}

\block{%
	\subsection{\mt{Építés}{}}
	
	\p{\mt{%
		A dokumentáció építéséhez a \code{compile} fájlt kell futtatni.
		Ez generált pdf fájlokat az \code{out} mappába helyezi.
	}{}}
}

