
\newpage
\section{Használat}

\block{%
	\subsection{Bejelentkezés}

	\p{\mt{%
			A program indításakor először a bejelentkező felületre fogad minket, ahol a bejelentkezéshez meg kell adnunk a felhasználónevet, a API szerver url -jét, a jelszót és a használni kívánt nyelvet (angol/magyar).
			A megfelelő adatok megadása után, a Bejelentkezés vagy Login gombra kattintva, amennyiben a megadott adatok helyesek, továbbjutunk a kezelőfelületre.
			A lap bal felső sarkában láthatjuk az alkalmazást használó személy nevét, illetve a mellette található gombal ki is jelentkezhetünk a programból.
			A lap tetején elhelyezkedő menüelemek segítségével választhatjuk ki, hogy az adatbázis mely részét szeretnénk megtekinteni vagy módosítani.
		}{}

		\ui{login.png}
	}
}

\block{%
	\subsection{Felhasználói fiókok kezelése}

	\p{\mt{%
			A Felhasználók menüpont segítségével megtekinthetjük az összes felhasználót, aki hozzáféréssel rendelkezik az alkalmazáshoz.
			Minden felhasználó mellett két gomb található, amelyek segítségével: Módosíthatjuk a felhasználó adatait (név, felhasználónév, jogosultságok, jelszó) vagy törölhetjük a felhasználót.
			A lap alján található gombokkal: Frissíthetjük az oldalt vagy új felhasználót regisztrálhatunk.
		}{}

		\ui{user.list.png}
	}
}

\block{%
	\subsubsection{Felhasználói fiókok hozzáadása}

	\p{\mt{%
			Felhasználói fiók hazzáadásához a \add gombra kell kattintanunk, majd a megjelenő bemeneti ablak kitöltése után, a \tick -ra kattintva tudjuk elmenteni azt.
		}{}

		\ui{user.add.png}
	}
}

\block{%
	\subsubsection{Felhasználói fiókok módosítása}

	\p{\mt{%
			A felhasználó adatait a \edit gombra való kattintással módosíthatjuk.
			Ezután a megjelent adatlap ki lesz töltve a mentett értékekkel.
			A módosítás után a \tick gombra kattintással lehet menteni.
		}{}

		\ui{user.edit.png}
	}
}

\block{%
	\subsubsection{Felhasználó engedélyeinek módosítása}

	\p{\mt{%
			A felhasználó engedélyeit a \authorizations gombra való kattintással módosíthatjuk.
			Ilyenkor feljön egy ablak amiben az összes engedély ki van listázva.
			Itt a \cross gombbal vehetünk el és a \tick gombbal adhatunk engedélyt.
		}{}

		\uib{authorization.system.png}
	}
}

\block{%
	\subsubsection{Felhasználói fiókok törlése}

	\p{\mt{%
			A felhasználót a \delete gombra való kattintással törölhetjük.
		}{}

		\ui{user.delete.png}
	}
}

\block{%
	\subsection{Egységek kezelése}

	\p{\mt{%
			Az Egységek menüpont segítségével megtekinthetjük, módosíthatjuk, illetve törölhetjük a nyilvántartásban szereplő, mérésre szolgáló egységeket. 
			A lap alján itt is megtalálhatóak a frissítés és a hozzáadás gombok, amelyekkel frissíthetjük az oldalt, illetve a megfelelő adatok megadásával újabb egységgel bővíthetjük az adatbázist.
		}{}

		\ui{unit.list.png}
	}
}

\block{%
	\subsubsection{Egységek hozzáadása}

	\p{\mt{%
			Egy egységet a \add gombra való kattintás után megjelenő bemeneti ablak kitöltésével, majd a \tick gombra való kattintással hozhatunk létre.
		}{}

		\ui{unit.add.png}
	}
}

\block{%
	\subsubsection{Egységek módosítása}

	\p{\mt{%
			Egységeket a \edit gombra való kattintással módosíthatunk. A kattintást követően megjelennek az adatok egy felugró ablakban, ahol könnyedén módosíthatjuk is azokat.
			A módosítás után a \tick gombra való kattintással menthetünk el a változtatásokat.
		}{}

		\ui{unit.edit.png}
	}
}

\block{%
	\subsubsection{Egységek törlése}

	\p{\mt{%
			Egységeket a \delete gombra való kattintással törölhetünk.
			Fontos megjegyezni, hogy egy egységet nem törölhetünk ha legalább egy cikk azt használja.
		}{}

		\ui{unit.delete.png}
	}
}

\block{%
	\subsection{Árucikkek kezelése}

	\p{\mt{%
			Az árucikkek menüpont segítségével megtekinthetjük, módosíthatjuk, illetve törölhetjük a nyilvántartásban szereplő, árucikkeket. 
			A lap alján itt is megtalálhatóak a frissítés és a hozzáadás gombok, amelyekkel frissíthetjük az oldalt, illetve a megfelelő adatok megadásával újabb cikkekel bővíthetjük a rendszert.
		}{}

		\ui{item.list.png}
	}
}

\block{%
	\subsubsection{Árucikkek hozzáadása}

	\p{\mt{%
			Egy árucikket a \add gombra való kattintás után megjelenő bemeneti ablak kitöltésével, majd a \tick gombra való kattintással hozhatunk létre.
		}{}

		\ui{item.add.png}
	}
}

\block{%
	\subsubsection{Árucikkek módosítása}

	\p{\mt{%
			Árucikkeket a \edit gombra való kattintással módosíthatunk. A kattintást követően megjelennek az adatok egy felugró ablakban, ahol könnyedén módosíthatjuk is azokat.
			A módosítás után a \tick gombra való kattintással menthetünk el a változtatásokat.
		}{}

		\ui{item.edit.png}
	}
}

\block{%
	\subsubsection{Árucikkek törlése}

	\p{\mt{%
			Árucikkeket a \delete gombra való kattintással törölhetünk. Fontos megjegyezni, hogy egy árucikket nem törölhetünk, ha az adott cikkből akár csak egy egységnyi is tárolva van a rendszerben.
		}{}

		\ui{item.delete.png}
	}
}

\block{%
	\subsection{Telephelyek kezelése}

	\p{\mt{%
			Az telephelyek menüpont segítségével megtekinthetjük, módosíthatjuk, illetve törölhetjük a nyilvántartásban szereplő, telephelyeket. 
			A lap alján itt is megtalálhatóak a frissítés és a hozzáadás gombok, amelyekkel frissíthetjük az oldalt, illetve a megfelelő adatok megadásával újabb telephellyel bővíthetjük a rendszert.
		}{}

		\ui{warehouse.list.png}
	}
}

\block{%
	\subsubsection{Telephelyek hozzáadása}

	\p{\mt{%
			Egy telephelyet a \add gombra való kattintás után megjelenő bemeneti ablak kitöltésével, majd a \tick gombra való kattintással hozhatunk létre.
		}{}

		\ui{warehouse.add.png}
	}
}

\block{%
	\subsubsection{Telephelyek módosítása}

	\p{\mt{%
			Telephelyeket a \edit gombra való kattintással módosíthatunk. A kattintást követően megjelennek az adatok egy felugró ablakban, ahol könnyedén módosíthatjuk is azokat.
			A módosítás után a \tick gombra való kattintással menthetünk el a változtatásokat.
		}{}

		\ui{warehouse.edit.png}
	}
}

\block{%
	\subsubsection{Telephelyek törlése}

	\p{\mt{%
			Telephelyet a \delete gombra való kattintással törölhetünk.
			Fontos megjegyezni, hogy amennyiben található benne raktár, a telephelyet nem törölhetjük, így először a benne található raktárakat kell törölni.
		}{}

		\ui{warehouse.delete.png}
	}
}

\block{%
	\subsection{Raktárak kezelése}

	\p{\mt{%
			A Raktárak menüpontra kattintva a program megmutatja számunkra a raktárakat telephelyenként szétválogatva.
			A menüsor alatt található lenyíló listával tudjuk kiválasztani, hogy melyik telephelyhez tartozó raktárakat szeretnénk megtekinteni, majd a raktárak neve mellett található gombokkal módosíthatjuk és törölhetjük azokat.
			A lap alján elhelyezkedő, hozzáadás gombbal pedig új raktárt adhatunk hozzá a kiválasztott telephelyhez.
		}{}

		\ui{storage.list.png}
	}
}

\block{%
	\subsubsection{Raktárak hozzáadása}

	\p{\mt{%
			Egy raktárt a \add gombra való kattintás után megjelenő bemeneti ablak kitöltésével, majd a \tick gombra való kattintással hozhatunk létre.
		}{}

		\ui{storage.add.png}
	}
}

\block{%
	\subsubsection{Raktárak módosítása}

	\p{\mt{%
			Raktárakat a \edit gombra való kattintással módosíthatunk. A kattintást követően megjelennek az adatok egy felugró ablakban, ahol könnyedén módosíthatjuk is azokat.
			A módosítás után a \tick gombra való kattintással menthetünk el a változtatásokat.
		}{}

		\ui{storage.edit.png}
	}
}

\block{%
	\subsubsection{Raktárak törlése}

	\p{\mt{%
			Raktárakat a \delete gombra való kattintással törölhetünk.
			Fontos megjegyezni, hogy egy raktárt nem törölhetünk addig, amíg található benne áru, tehát először ki kell üríteni a raktárt, hogy véghez tudjuk vinni a törlést.
		}{}

		\ui{storage.delete.png}
	}
}

\block{%
	\subsubsection{Raktár limitek}

	\p{\mt{%
			Raktár limiteket a \limit gombra való kattintással szerkeszthetünk.
		}{}

		\uib{storage.limit.png}
	}
}

\block{%
	\subsection{Műveletek}

	\subsubsection{Műveletek hozzáadása}

	\p{\mt{%
			Egy műveletet a \add gombra való kattintás után megjelenő bemeneti ablak kitöltésével, majd a \tick gombra való kattintással hozhatunk létre.
		}{}

		\uib{operation.add.png}
	}
}

\block{%
	\subsubsection{Műveletek törlése}

	\p{\mt{%
			Műveleteket a \delete gombra való kattintással törölhetünk.
		}{}

		\ui{operation.cancel.png}
	}
}

\block{%
	\subsubsection{Műveletek visszaigazolása}

	\p{\mt{%
			Műveleteket a \tick gombra való kattintással igazolhatunk vissza.
		}{}

		\ui{operation.commit.png}
	}
}

\block{%
	\subsection{Keresés}

	\p{\mt{%
			A Keresés használatával egyszerűen rá tudunk keresni bármilyen termékre. A termék nevének beírása után a program visszaadja, hogy melyik telephelynek, melyik raktárában található meg az adott cikk.
			Emellett további információkat is találhatunk az adott cikkről, mint például a rendelkezésre álló mennyiség vagy a termék sorozatszáma.
		}{}

		\ui{search.png}
	}
}

