
\newpage
\section{Használat}

\block{
	\subsection{Bejelentkezés}

	\p{\mt{%
		A program indításakor először a bejelentkező felületre fogad minket, ahol meg kell adni a felhasználónevet, a API szerver url -jét, a jelszót és nyelvet a bejelentkezéshez.
		Miután kitöltöttük az adatokat, a Bejelentkezés vagy Login gombra kattintva, amennyiben a megadott adatok helyesek, továbbjutunk a kezelőfelületre.
		A lap bal felső sarkában láthatjuk az alkalmazást használó személy nevét, illetve a mellette található gombal ki is jelentkezhetünk a programból. 
		A lap tetején elhelyezkedő menüelemek segítségével választhatjuk ki, hogy az adatbázis melyik részét szeretnénk megtekinteni vagy módosítani. 
		}{}

		\ui{login.png}
	}
}

\block{
	\subsection{Felhasználó kezelés}

	\p{\mt{%
		A Felhasználók menüpont segítségével megtekinthetjük az összes felhasználót, aki hozzáféréssel rendelkezik az alkalmazáshoz.
		Minden felhasználó mellett két gomb található, amelyek segítségével: 
		Módosíthatjuk a felhasználó adatait (név, felhasználónév, jogosultságok, jelszó) 
		Törölhetjük a felhasználót 
		A lap alján található gombokkal: 
		Frissíthetjük az oldalt 
		Új felhasználót regisztrálhatunk.
		}{}

		\ui{user.list.png}
	}
}

\block{
	\subsubsection{Felhasználó hozzáadás}

	\p{\mt{%
		Felhasználó hazzáadásához a \add gombra kell kattintani.
		Ezután megjelenik egy bemeneti ablak, ami kitöltése után a \tick gombra nyomhatunk a mentéshez.
		}{}

		\ui{user.add.png}
	}
}

\block{
	\subsubsection{Felhasználó módosítás}

	\p{\mt{%
		A felhasználó adatait a \edit gombra való kattintással módosíthatjuk.
		Ezután a megjelent adatlap ki lesz töltve a mentett értékekkel.
		A módosítás után a \tick gombra kattintással lehet menteni.
		}{}

		\ui{user.edit.png}
	}
}

\block{
	\subsubsection{Felhasználó engedélyek módosítása}

	\p{\mt{%
			A felhasználó engedélyeit a \authorizations gombra való kattintással módosíthatjuk.
			Ilyenkor feljön egy ablak amiben az összes engedély ki van listázva.
			Itt a \cross gombbal vehetünk el és a \tick gombbal adhatunk engedélyt.
		}{}

		\uib{authorization.system.png}
	}
}

\block{
	\subsubsection{Felhasználó törlés}

	\p{\mt{%
			A felhasználót a \delete gombra való kattintással törölhetjük.
		}{}

		\ui{user.delete.png}
	}
}

\block{
	\subsection{Egység kezelés}

	\p{\mt{%
			Az Egységek menüpont segítségével megtekinthetjük, módosíthatjuk, illetve törölhetjük a nyilvántartásban szereplő, mérésre szolgáló egységeket. 
			A lap alján itt is megtalálhatóak a frissítés és a hozzáadás gombok, amelyekkel frissíthetjük az oldalt, illetve a megfelelő adatok megadásával újabb egységgel bővíthetjük az adatbázist.
		}{}

		\ui{unit.list.png}
	}
}

\block{
	\subsubsection{Egység hozzáadás}

	\p{\mt{%
			Egy egységet a \add gombra való kattintással adhatunk hozzá a rendszerhez.
			Ekkor megjelenik egy bemeneti ablak amit kitöltve \tick gombra kattintással menthetünk.
		}{}

		\ui{unit.add.png}
	}
}

\block{
	\subsubsection{Egység módosítás}

	\p{\mt{%
			Egységeket a \edit gombra való kattintással módosíthatunk.
			Ekkor megjelennek az adatok egy felugró ablak és módosítás után a \tick gombra kattintással menthetünk.
		}{}

		\ui{unit.edit.png}
	}
}

\block{
	\subsubsection{Egység törlés}

	\p{\mt{%
			Egységeket a \delete gombra való kattintással törölhetünk.
			Fontos megjegyezni, hogy egy egységet nem törölhetünk ha legalább egy cikk azt használja.
		}{}

		\ui{unit.delete.png}
	}
}

\block{
	\subsection{Árucikkek kezelése}

	\p{\mt{%
			Az árucikkek menüpont segítségével megtekinthetjük, módosíthatjuk, illetve törölhetjük a nyilvántartásban szereplő, árucikkeket. 
			A lap alján itt is megtalálhatóak a frissítés és a hozzáadás gombok, amelyekkel frissíthetjük az oldalt, illetve a megfelelő adatok megadásával újabb cikkekel bővíthetjük a rendszert.
		}{}

		\ui{item.list.png}
	}
}

\block{
	\subsubsection{Árucikk hozzáadás}

	\p{\mt{%
			Egy árucikkek a \add gombbal adhatunk hozzá a rendszerhez.
			Ekkor megjelenik egy bemeneti ablak amit kitöltva \tick gombra kattintással menthetünk.
		}{}

		\ui{item.add.png}
	}
}

\block{
	\subsubsection{Árucikk módosítás}

	\p{\mt{%
			Árucikkeket a \edit gombra való kattintással módosíthatunk.
			Ekkor megjelennek az adatok egy felugró ablak és módosítás után a \tick gombra kattintással menthetünk.
		}{}

		\ui{item.edit.png}
	}
}

\block{
	\subsubsection{Árucikk törlés}

	\p{\mt{%
			Árucikkeket a \delete gombra való kattintással törölhetünk.
			Fontos megjegyezni, hogy egy árucikket nem törölhetünk ha van valahol ilyen tárolva a rendszerben.
		}{}

		\ui{item.delete.png}
	}
}

\block{
	\subsection{Telephely kezelés}

	\p{\mt{%
			Az telephelyek menüpont segítségével megtekinthetjük, módosíthatjuk, illetve törölhetjük a nyilvántartásban szereplő, telephelyeket. 
			A lap alján itt is megtalálhatóak a frissítés és a hozzáadás gombok, amelyekkel frissíthetjük az oldalt, illetve a megfelelő adatok megadásával újabb telephellyel bővíthetjük a rendszert.
		}{}

		\ui{warehouse.list.png}
	}
}

\block{
	\subsubsection{Telephely hozzáadás}

	\p{\mt{%
			Egy telephelyet a \add gombbal adhatunk hozzá a rendszerhez.
			Ekkor megjelenik egy bemeneti ablak amit kitöltva \tick gombra kattintással menthetünk.
		}{}

		\ui{warehouse.add.png}
	}
}

\block{
	\subsubsection{Telephely módosítás}

	\p{\mt{%
			Egy telephelyet a \edit gombra való kattintással módosíthatunk.
			Ekkor megjelennek az adatok egy felugró ablak és módosítás után a \tick gombra kattintással menthetünk.
		}{}

		\ui{warehouse.edit.png}
	}
}

\block{
	\subsubsection{Telephely törlés}

	\p{\mt{%
			Telephelyet a \delete gombra való kattintással törölhetünk.
			Fontos megjegyezni, hogy egy telephelyet nem törölhetünk ha van raktár benne.
			Azaz elöször a raktárakat kell törölni.
		}{}

		\ui{warehouse.delete.png}
	}
}

\block{
	\subsection{Raktár kezelés}

	\p{\mt{%
			A Raktárok menüpontra kattintva a program megmutatja számunkra a raktárokat telephelyenként szétválogatva.
			A menüsor alatt található lenyíló listával tudjuk kiválasztani, hogy melyik telephelyhez tartozó raktárokat szeretnénk megtekinteni, majd a raktárok neve mellett található gombokkal módosíthatjuk és törölhetjük azokat.
			A lap alján elhelyezkedő, hozzáadás gombbal pedig új raktárt adhatunk hozzá a kiválasztott telephelyhez. 
		}{}

		\ui{storage.list.png}
	}
}

\block{
	\subsubsection{Raktár hozzáadás}

	\p{\mt{%
			Egy telephelyet a \add gombbal adhatunk hozzá a rendszerhez.
			Ekkor megjelenik egy bemeneti ablak amit kitöltva \tick gombra kattintással menthetünk.
		}{}

		\ui{storage.add.png}
	}
}

\block{
	\subsubsection{Raktár módosítás}

	\p{\mt{%
			Egy raktárt a \edit gombra való kattintással módosíthatunk.
			Ekkor megjelennek az adatok egy felugró ablak és módosítás után a \tick gombra kattintással menthetünk.
		}{}

		\ui{storage.edit.png}
	}
}

\block{
	\subsubsection{Raktár törlés}

	\p{\mt{%
			Raktárakat a \delete gombra való kattintással törölhetünk.
			Fontos megjegyezni, hogy egy raktárt nem törölhetünk ha van benne áru.
			Azaz elöször ki kell üríteni a raktárt.
		}{}

		\ui{storage.delete.png}
	}
}

\block{
	\subsubsection{Raktár limitek}

	\p{\mt{%
			Raktár limiteket a \limit gombra való kattintással szerkeszthetünk.
		}{}

		\uib{storage.limit.png}
	}
}

\block{
	\subsection{Műveletek}

	\subsubsection{Művelet hozzáadás}

	\p{\mt{%
			Egy műveletet a \add gombbal adhatunk hozzá a rendszerhez.
			Ekkor megjelenik egy bemeneti ablak amit kitöltva \tick gombra kattintással menthetünk.
		}{}

		\uib{operation.add.png}
	}
}

\block{
	\subsubsection{Művelet törlés}

	\p{\mt{%
			Műveleteket a \delete gombra való kattintással törölhetünk.
		}{}

		\ui{operation.cancel.png}
	}
}

\block{
	\subsubsection{Művelet visszaigazolás}

	\p{\mt{%
			Műveleteket a \tick gombra való kattintással igazolhatunk vissza.
		}{}

		\ui{operation.commit.png}
	}
}

\block{
	\subsection{Keresés}

	\p{\mt{%
		A Keresés használatával egyszerűen rá tudunk keresni bármilyen termékre, a program pedig visszaadja, hogy melyik telephelynek melyik raktárában található meg az adott termék, illetve további információkat találunk, mint például a rendelkezésre álló mennyiség vagy a termék sorozatszáma. 
		}{}

		\ui{search.png}
	}
}

