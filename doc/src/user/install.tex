\newpage
\block{
	\section{Telepítés}

	\subsection{Server telepítés}

	\subsubsection{\mt{Linux mint}{}}

	\p{\mt{%
			A szerver telepítéséhez a futtatni kell a telepítő programot.
		}{}
		
		\cli{chmod +x install.sh \&\& ./install.sh}
		
		\mt{%
			Ez a parancs telepíti szükséges csomagokat és bemásolja a frontendet és a szervert a \code{/var/www/html/} mappába.
			Ezután ott be tudjuk állítani a szervert az \code{api/config.json} fájlban.
		}{}
	}
}

\block{
	\subsubsection{\mt{Más linux alapú rendszerek}{}}

	\p{\mt{%
			A szerver működéséhez a következő programok szükségesek:
		}{}
	}
}

\p{%
	\begin{itemize}
		\item php
		\item php-mysqli
		\item apache
		\item libapache2-mod-php
		\item mariadb-server
	\end{itemize}
}

\p{%
	\mt{%
		A szerver telepítéséhez az \code{api} és a \code{frontend/assets} mappát és a \code{frontend/index.html} fájlt át kell másolni az apache web mappába.
		A mysql szerveren létre kell hozni egy felhasználót és be kell állítani.
		Az apache szerveren engedélyezni kell a \code{.htaccess} fájlokat és be kell kapcsolni a \code{rewrite} és a \code{PHP} modulokat.
	}{}
}

\block{
	\subsubsection{Windows}

	\p{\mt{%
		A szerver Windowsra való telepítéséhez először a XAMPP-ot kell telepítenünk.
		Utána az \code{api} és a \code{frontend/assets} mappát és a \code{frontend/index.html} fájlt át kell másolni a htdocs mappába.
	}{}}
}

\block{
	\subsection{Asztali kliens telepítés}

	\p{\mt{%
		Az asztali kliens fájljai a \code{EasyInventoryDesktop/\allowbreak dist/\allowbreak boundles/\allowbreak EasyInventoryDesktop} mappában találhatóak.
		Ezt a mappát akárhova lehet másolni és nem igényel telepítőt, mert mined szükséges dll benne van a mappában.
		A futtatáshoz a benne található exe fájlt lehet futtatni.
	}{}}
}

\block{
	\subsection{Mobil kliens telepítés}

	\p{\mt{%
		A mobil alkalmazás telepítéséhez Android 14 szükséges.
	}{}}
}
