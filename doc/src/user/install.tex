\newpage
\section{Telepítés}

\block{%
	\subsection{Szerver telepítés}

	\subsubsection{\mt{Linux mint}{}}

	\p{\mt{%
			A szerver telepítéséhez le kell futtatni a telepítő programot:
		}{}
		
		\cli{chmod +x install.sh \&\& ./install.sh}
		
		\mt{%
			Ez a parancs telepíti a szükséges csomagokat, illetve bemásolja a frontendet és a
			szervert a \code{/var/www/html/} mappába. Ezt követően be tudjuk állítani a szervert az \code{api/config.json} fájlban.
		}{}
	}
}

\block{%
	\subsubsection{\mt{Más linux alapú rendszerek}{}}

	\p{\mt{%
			A szerver működéséhez a következő programok szükségesek:
		}{}
	}
}

\p{%
	\begin{itemize}
		\item php
		\item php-mysqli
		\item apache
		\item libapache2-mod-php
		\item mariadb-server
	\end{itemize}
}

\p{%
	\mt{%
		A szerver telepítéséhez az \code{api} és a \code{frontend/assets} mappát, illetve a \code{frontend/index.html} fájlt át kell másolni az apache web mappába.
		Ezt követően a mysql szerveren létre kell hozni egy felhasználót, majd beállítani azt.
		Végezetül pedig, az apache szerveren engedélyezni kell a \code{.htaccess} fájlokat, majd be kell kapcsolni a \code{rewrite} és a \code{PHP} modulokat.
	}{}
}

\block{%
	\subsubsection{Windows}

	\p{\mt{%
		A szerver Windowsra való telepítéséhez először a XAMPP-ot kell telepíteni, majd át kell másolni az \code{api} és a \code{frontend/assets} mappát, illetve a \code{frontend/index.html} fájlt a htdocs mappába.
	}{}}
}

\block{%
	\subsection{Asztali kliens telepítés}

	\p{\mt{%
		Az asztali kliens fájljai az \code{EasyInventoryDesktop/\allowbreak dist/\allowbreak boundles/\allowbreak EasyInventoryDesktop} mappában találhatóak.
		Ezt a mappát bárhová át lehet másolni és nem igényel telepítőt, mivel mined szükséges dll -t tartalmaz a mappában.
		Indításhoz a benne található exe fájlt kell futtatni.
	}{}}
}

\block{%
	\subsection{Mobil kliens telepítés}

	\p{\mt{%
		 A mobil kliens az Android Studió alátal létrehozott apk fájl futtatásával telepíthető Android 14-re.
	}{}}
}
