\newpage
\block{
    \section{Adminisztráció}

    \subsection{\mt{Arhivált adatok törlése}{}}

    \warning{\mt{%
	    Az arhivált adatok törlése visszafordíthatatlan!
	    Ezek az adatok adják a statisztikákat.
	    Régi adatok törlésével az akkori statisztika is törlődik.
    }{}}
}

\p{\mt{%
		Az arhivált adatok törlése hasznos lehet ha az adatbázis mérete már túl nagy.
		Az összes adat törlésére az alábbi parancsot kell futtatni
	}{}
	
	\cli{php archive delete all}
	
	\mt{%
		Általában nem akarjuk az összes arhivált adatot törölni.
		Ehez meg kell adni hány évnyi,heti vagy napi adatot akarunk megtartani az időegység jelével.
	}{}
	\begin{description}
		\item[d] \mt{nap}{}
		\item[w] \mt{hét}{}
		\item[y] \mt{év}{}
	\end{description}
	\mt{%
		Például: ez a parancs kitöli az összes arhivált adatot ami 5 évnél régebben lett törölve.
	}{}

	\cli{php archive delete 5y}
}

\block{
    \subsection{Eseménynapló kezelése}

    \p{\mt{%
	    Az eseménynapló sok hasznos információt tárol a szerver működéséről.
	    Ezekhez az információk segíthetnek a hibás beállítások javításán.
	    Az eseménynaplót a parancssoron keresztül kezelhetjük.
    }{}}
}

\block{
    \subsubsection{\mt{Eseménynapló megtekintése}{}}

    \p{\mt{%
		    Az eseménynapló megjelenítésére van a \code{log show} parancs.
			Ha nem adunk meg semmilyen kapcsolót akkor az összes adatot rövid formában írja ki.
	    }{}
	    
	    \cli{php log show}
    }
}

\p{\paragraph{-u,--user \mt{kapcsoló}{}}
	Ezzel megadhatjuk a felhasználói azonosítót amihez tartozó eseményeket akarjuk megtekinteni.
}

\p{\paragraph{-e, --error \mt{kapcsoló}{}}
	Szerverhibát okozó kéréseketre szűri le az eseményeket.
	Ezek lehetnek adatbázis kapcsolódási hibák.
}

\p{\paragraph{-r, --rejected \mt{kapcsoló}{}}
	Elutasított kérésekre szűri le az eseményeket.
}

\p{\paragraph{-b, --bad \mt{kapcsoló}{}}
	Hibás kérésekre szűri le az eseményeket.
}

\p{\paragraph{-s, --success \mt{kapcsoló}{}}
	Sikeres kérésekre szűri le az eseményeket.
}

\p{\paragraph{-v, --verbose \mt{kapcsoló}{}}
	Az eredményeket hosszú formában jeleníti meg.
}

\p{\paragraph{\mt{Pédák}{}}
	\mt{Megmutatja az összes kérést amit az \code{admin} felhasználó végzett}{}
	
	\cli{php log -u admin show}
	
	\mt{Megmutatja az összes szerverhibát és elutasított kérést}{}
	
	\cli{php log -er show}
	
	\mt{Megmutatja az összes szerverhibát részletesen}{}
	
	\cli{php log -ev show}
}

\block{
    \subsubsection{\mt{Egy esemény megtekintése}{}}

    \p{\mt{%
			Ha csak egy kérést akarunk megtekinteni akkor a kérés sorszámának megadásával tudjuk.
			Ehez a kéréshez nincsenek kapcsolók.
		}{}
		
		\cli{php log show 100}
	}
}

\block{
    \subsubsection{\mt{Eseménynapló törlése}{}}

    \p{\mt{%
		    Ez a parancs törli a teljes eseménynapló tartalmát:
	    }{}
	    
	    \cli{php log clear}
    }
}

\p{\mt{%
		Általában nem akarjuk az összes eseményt törölni.
		Ehez meg kell adni hány évnyi,heti vagy napi adatot akarunk megtartani az időegység jelével.
	}{}
	\begin{description}
		\item[d] \mt{nap}{}
		\item[w] \mt{hét}{}
		\item[y] \mt{év}{}
	\end{description}
	\mt{%
		Például: ez a parancs kitöli az összes eseményt ami 5 napnál régebbi.
	}{}
	\cli{php log clear 5d}
}

\block{
    \subsection{\mt{Felhasználó kezelés}{}}

    \p{\mt{%
	    A parancssorról a felhasználóknak új jelszót tudunk adni, felhasználókat tudunk törölni és új adminisztrátorokat létrehozni.
    }{}}
}

\block{
    \subsubsection{\mt{Adminisztrátor létrehozása}{}}

    \p{\mt{%
		    Új adminisztrátor létrehozásához meg kell adnunk az azonosítóját, a nevét és a jelszavát
	    }{}
	    
	    \cli{php user create admin Admin password123}
    }
}

\block{
    \subsubsection{\mt{Felhasználó törlése}{}}

    \p{\mt{%
		    Felhasználó törléséhez meg kell adnunk az azonosítóját.
		    Ez a művelet sikertelen ha az adott felhasználó más felhasználók főnöke.
	    }{}
	    
	    \cli{php user delete admin}
    }
}

\block{
    \subsubsection{\mt{Új jelszó adása}{}}

    \p{\mt{%
		    Az új jelszó adáshoz meh kell adni a felhasználó azonosítóját és az új jelszót.
	    }{}
	    
	    \cli{php user password admin password123}
    }
}

\block{
    \subsection{\mt{Engedélyek}{}}

    \p{\mt{%
	    Két féle engedély típus van: a rendszer és a helyi.
	    A rendszer engedélyek az egész rendszerre vonatkoznak.
	    Azaz ha egy felhasználó egy ilyen engedélyt kap akkor az minden telephelyen érvényes, de a rendszer engedélyek között rendszer adminisztrációval kapcsolatosak engedélyek is vannak.
	    A helyi engedélyek telephelyhez vannak kötve.
	    Azaz ha egy felhasználó egy ilyen engedélyt kap akkor csak az adott telephelyen érvényes.
    }{}}
}

\block{%
	\subsubsection{\mt{Rendszer engedélyek}{}}

	\p{%
		\begin{itemize}
			\item \mt{Telephelyek megtekintése}{}
			\item \mt{Telephelyek hozzáadása}{}
			\item \mt{Telephelyek módosítása}{}
			\item \mt{Telephelyek törlése}{}
			\item \mt{Típusok hozzáadása}{}
			\item \mt{Típusok módosítása}{}
			\item \mt{Típusok törlése}{}
			\item \mt{Felhasználók megtekintése}{}
			\item \mt{Felhasználók hozzáadása}{}
			\item \mt{Felhasználók módosítása}{}
			\item \mt{Felhasználók törlése}{}
		\end{itemize}
	}
}

\block{%
	\subsubsection{\mt{Helyi engedélyek}{}}

	\p{%
		\begin{itemize}
			\item \mt{Raktárak kezelése}{}
			\item \mt{Hozzáadó művelet létrehozása}{}
			\item \mt{Eltávolító művelet létrehozása}{}
			\item \mt{Műveletek visszaigazolása}{}
		\end{itemize}
	}
}

\block{%
	\subsection{\mt{Beállítások}{}}

	\p{\mt{%
		A \code{config.json} konfigurációs fájl a szerver beállításait tartalmazza.
	}{}}
}

\block{%
	\subsubsection{\mt{Adatbázis beállítások}{}}

	\p{%
		\mt{A \code{database} szekció az alkalmazás adatbázis-kapcsolatához szükséges paramétereket tartalmazza.}{}

		\begin{description}
			\item[host] \mt{Az adatbázis szerver címe.}{}
			\item[user] \mt{Az adatbázis felhasználó neve.}{}
			\item[password] \mt{Az adatbázis felhasználó jelszava.}{}
			\item[database\_name] \mt{Az adatbázis neve, amelyhez csatlakozik az alkalmazás (például: easyinventory).}{}
			\item[port] \mt{Az adatbázis portja, amelyen a kapcsolatot létesíteni kell.}{}
		\end{description}
	}
}

\block{%
	\subsubsection{\mt{Token lejárati idő}{}}

	\p{%
		\mt{A \code{token\_expiration} kulcs határozza meg, hogy hány másodpercig érvényes egy generált token. Ez az érték alapértelmezetten  120 másodperc, tehát 2 perc.}{}
	}
}

\block{%
	\subsubsection{\mt{Teszt token}{}}
	
	\p{%
		\mt{A \code{test\_token} kulcs egy fix token-t ad meg, amelyet a rendszer teszteléséhez használhatunk. Ezt normál futtatáskor \code{null}-ra kell állítani.}{}
	}
}

\block{%
	\subsubsection{\mt{Naplózási beállítások}{}}

	\p{%
		\mt{A \code{logging} szekció az alkalmazás különböző naplózási beállításait tartalmazza. Négy fő típusú naplózás van: \code{success}, \code{bad}, \code{rejected}, és \code{error}.
			Mindegyik típusnak van négy különböző beállítása a következő mezők szerint: \code{request\_body}, \code{response\_body}, \code{debug}, és \code{error}.}{}
	}
}

\p{%
	\begin{description}
		\item[success] \mt{Sikeres kérések naplózása}{}
		\item[bad] \mt{Hibás kérések naplózása}{}
		\item[rejected] \mt{Elutasított kérések naplózása}{}
		\item[error] \mt{Hiba naplózás}{}
	\end{description}
}

\p{%
	\begin{description}
		\item[request\_body] \mt{A kérés törzsének naplózása}{}
		\item[response\_body] \mt{A válasz törzsének naplózása}{}
		\item[debug] \mt{A hibakeresési információk naplózása}{}
		\item[error] \mt{A hibák naplózása}{}
	\end{description}
}

