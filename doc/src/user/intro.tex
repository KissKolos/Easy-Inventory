
\section{\mt{Bevezetés}{}}

\block{%
	\subsection{\mt{Fogalmak}{}}

	\subsubsection{Telephelyek}

	\p{\mt{%
		A telephelyek különböző telkek vagy raktárépületek, amelyek területén különböző raktárak találhatóak.
	}{}}
}

\block{%
	\subsubsection{Raktárak}
	
	\p{\mt{%
		A raktárakon belül találhatóak a különböző termékek, így a raktár számít a legkisseb tároló egységnek. A raktárak számára meg lehet adni, hogy mennyi fér beléjük az egyes cikkekből.
	}{}}
}

\block{%
	\subsubsection{Műveletek}

	\p{\mt{%
			A műveletek segítségével tudunk cikkeket hozzáadni vagy éppen eltávolítani a raktárakból. Mint arra a korábbi mondat is utalt, két féle művelet létezik: a hozzáadási és az eltávolítási.
		}{}
	}
}

\p{%
	\paragraph{\mt{Hozzáadási műveletek}{}}
	\mt{%
		A hozzáadási művelet használatával különböző cikkeket tudunk felvinni egy adott raktár nyilvántartásába. Létrehozásakor lehetőségünk van különböző követelmények megadására is, mellyekkel szabályozhatjuk, hogy hová kerülnek a hozzáadásra szánt cikkek.
	}{}
}

\p{%
	\paragraph{\mt{Eltávolítási műveletek}{}}
	\mt{%
		Az eltávolítási művelet segítségével lehetőségünk van cikkek kivételére a hozzájuk tartozó raktárból. Létrehozásakor a hozzáadási művelethez hasonló képpen itt is lehetőségünk van szabályozni, hogy melyik raktárból és melyik cikkek kerüljenek eltávolításra, amit meghatározhatunk lot-, vagy akár sorozatszám megadásával is.
	}{}
}


\block{%
	\subsubsection{Lot számok}
	
	\p{\mt{%
		A lot számok különböző cikk csoportok megkülönböztetésére szolgáló kódok, amelyek nagyobb nyílvántartások kezelése esetén nélkülözhetetlenek.
	}{}}
}

\block{%
	\subsubsection{Globális sorozat szám}
	
	\p{\mt{%
		A globális sorozatszám egy olyan szám, amelynek az egész rendszeren belül egyedinek kell lennie, még az azonos árucikkek között is.
	}{}}
}

\block{%
	\subsubsection{Gyártói sorozat szám}
	
	\p{\mt{%
		Minden hozzáadott árucikkhez hozzárendelhetünk egy gyártói sorozatszámot.
		Ezeknek a számoknak nem feltétlenül kell egyedinek lenniük, mivel több gyártó esetén, az árucikkek sorozatszámai akár ütközhetnek is.
	}{}}
}
