\lstset{language=Java}

\block{%
	\section{\mt{Asztali alkalmazás}{}}

	\subsection{\mt{Áttekintés}{}}
	
	\p{\mt{%
		Az asztali alkalmazás a TestNG tesztelési keretrendszert használja.
	}{}}
}

\block{%
	\subsection{\mt{E2E tesztek}{}}
	
	\p{\mt{%
		Az asztali alkalmazás tesztelése olyan E2E teszteket történik, amelyek felhasználói interakciókat szimulálnak és megjelenített szövegeket ellenőriznek. A felhasználói felület elemeinek tesztelhetősége érdekében az elemek CSS
osztályokkal vannak ellátva. A tesztek egy része képernyőképeket készít a dokumentációhoz.
	}{}}
}

\block{%
	\subsection{\mt{NodeMatcher osztály}{}}
	
	\p{\mt{%
			A \code{NodeMatcher} osztály feladata a felhasználói felület elemeinek kiválasztása és az interakció szimulálása a kiválasztott elemekkel.	
		}{}

		\lstinputlisting{generated/desktop.nodematcher.java}
	}
}

\block{%
	\subsubsection{allWindowRoots}

	\p{\mt{%
			Az \code{allWindowRoots} funkció létrehoz egy \code{NodeMatcher} objektumot amiben minden ablak gyökér eleme ki van választva.
		}{}
	}
}


\block{%
	\subsubsection{descendants}

	\p{\mt{%
			A \code{descendants} metódus létrehoz egy \code{NodeMatcher} objektumot amiben azok az elemek vannak kijelölve amik legalább egy kijelölt elem leszármazottja.
		}{}
	}
}

\block{%
	\subsubsection{with}

	\p{\mt{%
			A \code{with} metódus bekér egy lambda függvényt, amellyel egy \code{NodeMatcher} objektumot egy másik \code{NodeMatcher} objektummá alakít át.
			Erre a folyamatra azért van szükség, hogy a metódus le tudja szűrni a kijelölt elemeket és elérni, hogy csak olyan kiválasztást adjon vissza, amelyben legalább egy elem található.
			A függvény bemenete minden estben egy olyan kijelölés, amiben megtalálható az az elem, amit éppen szűr.
			A leszűrés végén a metódus a kijelölést egy új \code{NodeMatcher} objektum formájában kiadja.
			Ez például arra használható ha csak azokat az elemeket akarjuk amikben van egy bizonyos címke.
		}{}
	}
}

\block{%
	\subsubsection{withClass}

	\p{\mt{%
			A \code{withClass} metódus egy olyan \code{NodeMatcher} objektumot hoz létre, amelyben csak azok az elemek vannak kiválasztva, amellyek el vannak látva a megadott CSS stílussal.
		}{}
	}
}

\block{%
	\subsubsection{click}

	\p{\mt{%
			A \code{click} metódus kattintást szimulál a kiválasztott gombon.
			Ha nem gomb van kiválasztva vagy nem egy elem van kiválasztva akkor hibát dob.
		}{}
	}
}

\block{%
	\subsubsection{replaceText}

	\p{\mt{%
			A \code{replaceText} metódus szöveg beírását szimulálja a kiválasztott bemenetí mezőn.
			Ha nem bemeneti mező van kiválasztva vagy nem egy elem van kiválasztva akkor hibát dob.
		}{}
	}
}

\block{%
	\subsubsection{screenshotWindow}

	\p{\mt{%
			A \code{screenshotWindow} metódus a kijelölt elem ablakáról készít egy képernyőképet és a megadott névvel elmenti.
			Ha egy elem van kiválasztva akkor hibát dob.
		}{}
	}
}

